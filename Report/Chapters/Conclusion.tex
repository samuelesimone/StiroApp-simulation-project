\chapter{Conclusion}\label{ch:conclusion}
The purpose of this agent-based simulation was to validate a startup idea by going to simulate what are the possible scenarios so as to understand if the idea could actually be brought to market. From what can be gleaned, the business model presented is rather weak, in the sense that, it is a very good starting point to start making money. However, in order for the startup to cover the costs of maintaining infrastructure as well as personnel, it is necessary to complement this model with another one. In fact, the ideas could be:
\begin{itemize}
 \item Insertion of non-invasive advertising, between posts.
 \item In-app section of e-shop intended both for Workers for cleaning garments as well as for Buyers for maintaining garments.
 \item In-app purchase for premium packages (quality or speed) of the service.
 \end{itemize} 
In fact, what can be seen from the various experiments is that the cost of riders can be high and therefore a way must be sought through which maintaining riders is not such an intrusive problem. 
 Also, in the simulation approach, some simplifications were made for the purpose of getting straight to the point and trying to validate the logic behind the model. For example, the cost of clothes is assigned by a discrete uniform random variable where the parameter b is 20. This number could represent an underestimate of the model. So already by increasing that parameter the gains could grow.
 Certainly, both market analysis and business model analysis require a deeper and more detailed study by experts in the field. 
 In contrast, a simulated approach can give an idea of the feasibility of the service and experiment with different scenarios so as to maximize earnings.