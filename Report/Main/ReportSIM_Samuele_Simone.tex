

% !TeX spellcheck = en_US 
\documentclass[12pt,english]{report}
\usepackage{tesi}
% CORSO DI LAUREA:
\def\myCDL{Master in\\Computer Science}

% TITOLO REPORT:
\def\myTitle{Simulation \\
\large{Final report on Agent-Based Simulation for StiroApp}}

% AUTORE:
\def\myName{Samuele Simone}
\def\myMat{Matr. Nr. 11910A}

\def\myRefereeA{Prof. Alberto Ceselli}

% ANNO ACCADEMICO
\def\myYY{2022-2023}

% Il seguente comando introduce un elenco delle figure dopo l'indice (facoltativo)
%\figurespagetrue

% Il seguente comando introduce un elenco delle tabelle dopo l'indice (facoltativo)
%\tablespagetrue


% Package di formato
\usepackage[a4paper]{geometry}		% Formato del foglio
\usepackage[english]{babel}			% Supporto per l'italiano
\usepackage[utf8]{inputenc}			% Supporto per UTF-8
\usepackage[a-1b]{pdfx}			% File conforme allo standard PDF-A (obbligatorio per la consegna)

% Package per la grafica
\usepackage{graphicx}				% Funzioni avanzate per le immagini
\usepackage{hologo}					% Bibtex logo with \hologo{BibTeX}
%\usepackage{epsfig}				% Permette immagini in EPS
\usepackage{listings}
\usepackage{xcolor}
\usepackage{hyperref}

%Creating dark code theme for listings
\definecolor{codegreen}{rgb}{0.58,0.88,0.58}
\definecolor{codegray}{rgb}{0.5,0.5,0.5}
\definecolor{codeorange}{rgb}{0.72,0.54,0.45}
\definecolor{backcolour}{rgb}{0.10,0.13,0.14}
\definecolor{myorange}{RGB}{245,156,74}
\definecolor{keyw}{rgb}{0.60,0.85,0.98}
\lstdefinestyle{mystyle}{
    backgroundcolor=\color{backcolour},   
    commentstyle=\color{codegreen},
    keywordstyle=\color{keyw},
    numberstyle=\tiny\color{codegray},
    stringstyle=\color{codeorange},
    basicstyle=\ttfamily\footnotesize \color{white},
    breakatwhitespace=false,         
    breaklines=true,                 
    captionpos=b,                    
    keepspaces=true,                 
    numbers=left,                    
    numbersep=5pt,                  
    showspaces=false,                
    showstringspaces=false,
    showtabs=false,                  
    tabsize=2
}

\lstset{style=mystyle}

% Package tipografici
\usepackage{amssymb,amsmath,amsthm} % Simboli matematici
\usepackage{listings}				% Scrittura di codice

% Package ipertesto
\usepackage{url}					% Visualizza e rendere interattii gli URL
\usepackage{hyperref}				% Rende interattivi i collegamenti interni
\usepackage{notes2bib}

\usepackage{multirow}
\hypersetup{
    colorlinks=true,
    linkcolor=blue,
    urlcolor=myorange,
    filecolor=magenta,  
    }

\begin{document}

% Creazione automatica del frontespizio
\frontespizio
\beforepreface
\afterpreface


\chapter{Introduction}
The idea behind this project was to try to validate a startup idea by making use of simulation models. Specifically by leveraging an agent-based simulation.
In the report therefore we will go over what StiroApp is, how it works, the models that have been used as well as the technical implementation. Through the use of KPIs we will study the performance indicators of the model. After that there will be a chapter dedicated to experiments and finally the conclusions that can be drawn from it.

\section{StiroApp - The story behind}
StiroApp was born from the desire to bring to the market of applications, and more generally of services, a system that can connect two categories of people: Workers and Buyers. In more detail, the service seeks to make the process of ironing/washing one's clothing faster, more streamlined and more economical. I started by analyzing people's desire/problems by going to create user-stories necessary to make the product as user-centric as possible. As an example I reported one user story for a better understanding of the problem.
\subsection{User story - Max}
\begin{itemize}
\item \textbf{Context}: Max is a 27-year-old boy. He has been living alone for about a year. He moved to Milan for work reasons. He's a video game enthusiast.
\item \textbf{Problems:} 
	\begin{itemize}
		\item Because of his busy life, He does not have time to iron/wash his clothes
		\item It is a time-consuming, tedious and even difficult activity
	\end{itemize}
\end{itemize}
\section{Understanding the problem - More deeply}
After a series of dutiful analyses and interviews at the front to examine the problem as best as possible here is what can be extracted in a more abstract way representing the basic principles for which is Stirapp should be born.
\begin{itemize}
\item Only a few millennials make use of laundries
\item Daily laundry/ironing cannot be delegated to laundries
\item This type of daily activity (ironing/washing) is generally carried out by generation X (1965 - 1980) as opposed to generation Y (1980 - 1994) and Z (1995 - 2010)
\item Delivery/collection times of clothes are high
\end{itemize}
\section{Proposed solution}
The approach taken to solve these problems was therefore to offer an application for people in such a way as to connect the two categories into which we can classify users:
\begin{itemize}
\item Workers: Those who intend to use the application for the purpose of earning money by working for Buyers, thus ironing or washing their clothes.
\item Buyers: Those who intend to use the app as a service where they can get their laundry cleaned and ironed.
\end{itemize}
\subsection{Starting features}
The app can be divided into two major features:
\begin{itemize}
\item Create posts for your own clothes by specifying the type of service you want (example: washing, ironing, both) and your needs.
\item Wash, iron other users' clothes in such a way as to earn money.
\end{itemize}
\subsection{App Flow}
Below I show how the app should work. As an example during the agent-based simulation, some changes were made so as to focus more on the important parts of the app such as post creation or the worker system.
\begin{figure}[hbtp]
\caption{App flow Schema}
\centering
\includegraphics[scale=0.2]{../Images/AppFlow.png} 
\end{figure}
\begin{itemize}
\item \textbf{Utente 1 (Buyers)} enter in the app after the onboarding in which he can understand the main features of the app
\item \textbf{In-app onboarding} is the process of teaching users how to use an app to achieve their goals
\item \textbf{Signup-Signin}. The user must be logged or registered into the system before using the app 
\item The user will be redirect into the Main Screen of the app. Then with a bottom navigation bar he can choose in which page enter such us Notification screen, Account screen and the Create Post screen.
\item \textbf{Bid system} is a hypothetical part of the app in development
\item \textbf{Checkout post screen} is the confirm page after the accepted price
\item \textbf{Delivery tracking} is an important page for the user in order to check where is his order
\item \textbf{Utente 2} he is the Worker that is allowed to do the laundry or offer other type of services specified in the app during the create post screen
\end{itemize}
\subsection{UX and UI for StiroApp}
Before we get our hands on the code, it is important to study the user experience (UX) and user interface (UI) to finalize the creation of a user-friendly application.
So here I report some images of the work done:
\begin{figure}[hbtp]
\caption{Representation of the main features of the application through UX}
\centering
\includegraphics[scale=0.5]{../Images/ux.png}
\end{figure}
\begin{figure}[hbtp]
\caption{An example of Card UI for the post creation}
\centering
\includegraphics[scale=0.5]{../Images/cardui.png}
\end{figure}
\subsection{Mockup}
To give a real representation of StirApp to the reader I created a mockup. According to Decode Agency \cite{decodeagency}, an app mockup is a detailed representation of your app design. It contains all the final UI elements such as typography, copy, colors, and visuals like icons and photos. Sample content will also be used, but dummy text is also acceptable.
\begin{figure}[hbtp]
\caption{Main Feed Mockup}
\centering
\includegraphics[scale=0.1]{../Images/mockup02.png}
\end{figure}
\begin{figure}[hbtp]
\caption{Detail Item Screen Mockup}
\centering
\includegraphics[scale=0.1]{../Images/mockup01.jpg}
\end{figure}
\section{A look into the market}
By doing a market analysis I obtained those main pillars:
\begin{itemize}
\item Buyers find this activity (washing/draining) time-consuming
\item Workers find this activity (washing/ironing) necessary and  have the skills to do it
\item Buyers prefer to find  someone who can take care their clothes
\item Workers make a profit for fulfilling these tasks
\end{itemize}
\section{Business Model}
\chapter{Model}
In order to simulate the traffic over the app and to estimate revenues, costs and other KPI that are described in the Chapter \ref{ch:KPI}, I used the agent-based modeling.
As well described from the Columbia University website \cite{columbia}, Agent-based models are computer simulations used to study the interactions between people, things, places, and time. They are stochastic models built from the bottom up meaning individual agents (often people in epidemiology) are assigned certain attributes.The agents are programmed to behave and interact with other agents and the environment in certain ways. \par
The agents that are involved in the model are:
\begin{itemize}
\item \textbf{Worker\_iron}: Represents the worker within the application
\item \textbf{Buyer}: Represents the buyer, i.e., the one who creates the posts with their clothing
\item \textbf{Rider}: He is in charge of picking up the goods and delivering them to the respective worker/buyer.
\item \textbf{Scheduler}: He/she is in charge of handling incoming posts and directing them
\end{itemize}
We will see during this Chapter how I modeled these agents for StiroApp.
\section{Worker\_iron agent}
Below I will go on to describe the logic of the Worker\_iron agent. The technical details will be discussed in Chapter \ref{ch:implementation}.
Basically the worker as soon as it is generated waits for some order to be assigned to it, in a \textbf{Waiting state}. 
Through a branch the worker by condition wonders whether it has received the order or not. In the first case then he can start working by entering the \textbf{Working state} and after a Uniform Discrete random variable $\mathcal{U}(a,b)$ where $a = 2, b = 60$ minutes, he enter in a \textbf{ReadyForDelivery state} Then, with a 10 minutes Timeout transition enter in a branch and through a Bernoulli random variable $\mathcal{B}(p)$ where $p = 0.7$ he decides whether to finish his task or re-enter the \textbf{Waiting state}.In the second case, on the other hand, he enters the \textbf{Thinking state} and then after a small timeout point to the same transition branch where there is the Bernoulli random variable.
\section{Buyer agent}
The Buyer as soon as it is generated is in a \textbf{Waiting state}. In fact here, through a transition to the branch you want to check if that specific agent is new or already created and therefore is waiting for his order. If he is new then he can create a post with his clothing by entering the \textbf{CreatePost state}. After that through a Timeout of 10 min he enters the \textbf{Waiting state} and also through a Bernoulli random variable $\mathcal{B}(p)$ where $p = 0.7$ he decides whether to finish his task or re-enter the \textbf{Waiting state}. If,instead, he has already a pending order and it's notified as delivered he can again through the same branch if finish or re-enter.
\section{Rider agent}
The rider plays a key role within my simulation. In fact it involves a slightly more complex logic in that its task depends on the type of order received. Initially it is in a \textbf{Waiting state}. As soon as it receives an order it goes into the \textbf{AssignedOrder state}. Then it has to figure out through the properties of the received order whether it is a pickup or a delivery. In fact the rider first has to go to the buyer and receive the clothes for which it wants to perform a service and from there, the rider has to deliver it at the chosen worker. Once this is done through a Bernoulli random variable $\mathcal{B}(p)$ where $p = 0.7$ decides whether to continue working enter the Waiting state again or stop working. If it continues working it may happen to receive the delivery order and then go to a worker who is in the ReadyForDelivery state, pick up the clothes, and return them to the buyer who owns those clothes.
\section{Scheduler agent}
This agent, on the other hand, acts as a conduit between agents by handling the exchange of messages (specifically, we will see that we will be dealing with the dispatch of an object of type Post). It presents two states, a \textbf{Wait state} and a \textbf{Dispatch state}. It simply presents two transitions and allow you to manage the event queue and based on the type of post received sort the dispatches to the correct recipient.
\chapter{Implementation}\label{ch:implementation}
ci
\chapter{KPI}\label{ch:KPI}
ci
\chapter{Experiments} \label{ch:experiments}
In this Chapter we will explore different scenarios obtained by changing parameters, number of population agents and we will try to maximize the Revenue as the KPI of the system. For the note, comments see the Chapter \ref{ch:conclusion}.
\subsection{Original setup}
From the simulation panel I setup the simulation duration to 960 minutes that correspond to 16 hours. The initial population are:
\begin{itemize}
\item 20 Worker\_iron
\item 15 Buyer
\item 10 Rider
\end{itemize}
Then the other parameters:
\begin{itemize}
\item perc\_revenue : 0.15\%
\item fixed\_cost :0.20€
\item workerArrRate: 20 per hour
\item buyerArrRate: 20 per hour
\item riderArrRate: 10 per day
\item Cost: 64€ per day each rider
\end{itemize}
Here the results:
\begin{figure}[hbtp]
\caption{Original setup simulation}
\centering
\includegraphics[scale=0.3]{../Images/sim01.png}
\end{figure}
As you can see from the top KPI or form the chart in the bottom-right side the Revenue are below the Cost. So there will be a lost over the first day. So we need to change some parameters and check if the situation will be better or not. For doing that I applied the What-If scenario different time as reported below.
\subsection{What-If scenario 1: Same configuration - lower rider cost}
By just reducing the cost of the rider we can obtain a profit, with the same parameters. Of course this is not the optimal way. Indeed there are national law that ensure the right price for each worker, in this case the rider. However here the simulation image:
\begin{figure}[hbtp]
 \caption{Lower price for rider}
 \centering
 \includegraphics[scale=0.3]{../Images/sim03.png}
 \end{figure}
\subsection{What-If scenario 2: Increasing earning parameters}
Configuration setup:
\begin{itemize}
\item perc\_revenue : 0.3\%
\item fixed\_cost :0.30€
\item workerArrRate: 20 per hour
\item buyerArrRate: 20 per hour
\item riderArrRate: 0 per day
\item Cost: 64€ per day each rider
\end{itemize}
Here the result:
\begin{figure}[hbtp]
\caption{Increasing earning params simulation}
\centering
\includegraphics[scale=0.3]{../Images/sim04.png}
\end{figure}
\subsection{What-If scenario 3: Increasing rider speed}
Originally, the rider speed was set to 10 km/h. By increasing this number to 50 km/h the percentage of valid transaction is 52\% more then the previous experiment. Then, for this experiment, the initial number of riders are 100. 
The result are shown in Figure \ref{figure:sim05}.
\begin{figure}[hbtp]
\caption{Increasing  rider speed simulation}
\label{figure:sim05}
\centering
\includegraphics[scale=0.3]{../Images/sim05.png}
\end{figure}
\subsection{What-If scenario 4: Increasing worker and buyer rate (50 per hour)}
By increasing the number of order and the number of worker to 50 per hour we obtain 4200€ as Revenue with 276 valid transaction. The cost still remain 3200 considering 100 rider as amount of population. 
The result are shown in Figure \ref{figure:sim06}.
\begin{figure}[hbtp]
\caption{Increasing number of worker/buyer per hour }
\centering
\label{figure:sim06}
\includegraphics[scale=0.3]{../Images/sim06.png}
\end{figure}

\chapter{Conclusion}\label{ch:conclusion}
The purpose of this agent-based simulation was to validate a startup idea by going to simulate what are the possible scenarios so as to understand if the idea could actually be brought to market. From what can be gleaned, the business model presented is rather weak, in the sense that, it is a very good starting point to start making money. However, in order for the startup to cover the costs of maintaining infrastructure as well as personnel, it is necessary to complement this model with another one. In fact, the ideas could be:
\begin{itemize}
 \item Insertion of non-invasive advertising, between posts.
 \item In-app section of e-shop intended both for Workers for cleaning garments as well as for Buyers for maintaining garments.
 \item In-app purchase for premium packages (quality or speed) of the service.
 \end{itemize} 
In fact, what can be seen from the various experiments is that the cost of riders can be high and therefore a way must be sought through which maintaining riders is not such an intrusive problem. 
 Also, in the simulation approach, some simplifications were made for the purpose of getting straight to the point and trying to validate the logic behind the model. For example, the cost of clothes is assigned by a discrete uniform random variable where the parameter b is 20. This number could represent an underestimate of the model. So already by increasing that parameter the gains could grow.
 Certainly, both market analysis and business model analysis require a deeper and more detailed study by experts in the field. 
 In contrast, a simulated approach can give an idea of the feasibility of the service and experiment with different scenarios so as to maximize earnings.

%
%			BIBLIOGRAFIA
%

\bibliographystyle{unsrt}
\bibliography{bibliografia}
\addcontentsline{toc}{chapter}{References}


\end{document}


 
